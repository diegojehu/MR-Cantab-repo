% Options for packages loaded elsewhere
\PassOptionsToPackage{unicode}{hyperref}
\PassOptionsToPackage{hyphens}{url}
%
\documentclass[
]{article}
\usepackage{amsmath,amssymb}
\usepackage{lmodern}
\usepackage{iftex}
\ifPDFTeX
  \usepackage[T1]{fontenc}
  \usepackage[utf8]{inputenc}
  \usepackage{textcomp} % provide euro and other symbols
\else % if luatex or xetex
  \usepackage{unicode-math}
  \defaultfontfeatures{Scale=MatchLowercase}
  \defaultfontfeatures[\rmfamily]{Ligatures=TeX,Scale=1}
\fi
% Use upquote if available, for straight quotes in verbatim environments
\IfFileExists{upquote.sty}{\usepackage{upquote}}{}
\IfFileExists{microtype.sty}{% use microtype if available
  \usepackage[]{microtype}
  \UseMicrotypeSet[protrusion]{basicmath} % disable protrusion for tt fonts
}{}
\makeatletter
\@ifundefined{KOMAClassName}{% if non-KOMA class
  \IfFileExists{parskip.sty}{%
    \usepackage{parskip}
  }{% else
    \setlength{\parindent}{0pt}
    \setlength{\parskip}{6pt plus 2pt minus 1pt}}
}{% if KOMA class
  \KOMAoptions{parskip=half}}
\makeatother
\usepackage{xcolor}
\IfFileExists{xurl.sty}{\usepackage{xurl}}{} % add URL line breaks if available
\IfFileExists{bookmark.sty}{\usepackage{bookmark}}{\usepackage{hyperref}}
\hypersetup{
  pdftitle={Practical 1},
  pdfauthor={Diego Aguilar-Ramirez},
  hidelinks,
  pdfcreator={LaTeX via pandoc}}
\urlstyle{same} % disable monospaced font for URLs
\usepackage[margin=1in]{geometry}
\usepackage{color}
\usepackage{fancyvrb}
\newcommand{\VerbBar}{|}
\newcommand{\VERB}{\Verb[commandchars=\\\{\}]}
\DefineVerbatimEnvironment{Highlighting}{Verbatim}{commandchars=\\\{\}}
% Add ',fontsize=\small' for more characters per line
\usepackage{framed}
\definecolor{shadecolor}{RGB}{248,248,248}
\newenvironment{Shaded}{\begin{snugshade}}{\end{snugshade}}
\newcommand{\AlertTok}[1]{\textcolor[rgb]{0.94,0.16,0.16}{#1}}
\newcommand{\AnnotationTok}[1]{\textcolor[rgb]{0.56,0.35,0.01}{\textbf{\textit{#1}}}}
\newcommand{\AttributeTok}[1]{\textcolor[rgb]{0.77,0.63,0.00}{#1}}
\newcommand{\BaseNTok}[1]{\textcolor[rgb]{0.00,0.00,0.81}{#1}}
\newcommand{\BuiltInTok}[1]{#1}
\newcommand{\CharTok}[1]{\textcolor[rgb]{0.31,0.60,0.02}{#1}}
\newcommand{\CommentTok}[1]{\textcolor[rgb]{0.56,0.35,0.01}{\textit{#1}}}
\newcommand{\CommentVarTok}[1]{\textcolor[rgb]{0.56,0.35,0.01}{\textbf{\textit{#1}}}}
\newcommand{\ConstantTok}[1]{\textcolor[rgb]{0.00,0.00,0.00}{#1}}
\newcommand{\ControlFlowTok}[1]{\textcolor[rgb]{0.13,0.29,0.53}{\textbf{#1}}}
\newcommand{\DataTypeTok}[1]{\textcolor[rgb]{0.13,0.29,0.53}{#1}}
\newcommand{\DecValTok}[1]{\textcolor[rgb]{0.00,0.00,0.81}{#1}}
\newcommand{\DocumentationTok}[1]{\textcolor[rgb]{0.56,0.35,0.01}{\textbf{\textit{#1}}}}
\newcommand{\ErrorTok}[1]{\textcolor[rgb]{0.64,0.00,0.00}{\textbf{#1}}}
\newcommand{\ExtensionTok}[1]{#1}
\newcommand{\FloatTok}[1]{\textcolor[rgb]{0.00,0.00,0.81}{#1}}
\newcommand{\FunctionTok}[1]{\textcolor[rgb]{0.00,0.00,0.00}{#1}}
\newcommand{\ImportTok}[1]{#1}
\newcommand{\InformationTok}[1]{\textcolor[rgb]{0.56,0.35,0.01}{\textbf{\textit{#1}}}}
\newcommand{\KeywordTok}[1]{\textcolor[rgb]{0.13,0.29,0.53}{\textbf{#1}}}
\newcommand{\NormalTok}[1]{#1}
\newcommand{\OperatorTok}[1]{\textcolor[rgb]{0.81,0.36,0.00}{\textbf{#1}}}
\newcommand{\OtherTok}[1]{\textcolor[rgb]{0.56,0.35,0.01}{#1}}
\newcommand{\PreprocessorTok}[1]{\textcolor[rgb]{0.56,0.35,0.01}{\textit{#1}}}
\newcommand{\RegionMarkerTok}[1]{#1}
\newcommand{\SpecialCharTok}[1]{\textcolor[rgb]{0.00,0.00,0.00}{#1}}
\newcommand{\SpecialStringTok}[1]{\textcolor[rgb]{0.31,0.60,0.02}{#1}}
\newcommand{\StringTok}[1]{\textcolor[rgb]{0.31,0.60,0.02}{#1}}
\newcommand{\VariableTok}[1]{\textcolor[rgb]{0.00,0.00,0.00}{#1}}
\newcommand{\VerbatimStringTok}[1]{\textcolor[rgb]{0.31,0.60,0.02}{#1}}
\newcommand{\WarningTok}[1]{\textcolor[rgb]{0.56,0.35,0.01}{\textbf{\textit{#1}}}}
\usepackage{graphicx}
\makeatletter
\def\maxwidth{\ifdim\Gin@nat@width>\linewidth\linewidth\else\Gin@nat@width\fi}
\def\maxheight{\ifdim\Gin@nat@height>\textheight\textheight\else\Gin@nat@height\fi}
\makeatother
% Scale images if necessary, so that they will not overflow the page
% margins by default, and it is still possible to overwrite the defaults
% using explicit options in \includegraphics[width, height, ...]{}
\setkeys{Gin}{width=\maxwidth,height=\maxheight,keepaspectratio}
% Set default figure placement to htbp
\makeatletter
\def\fps@figure{htbp}
\makeatother
\setlength{\emergencystretch}{3em} % prevent overfull lines
\providecommand{\tightlist}{%
  \setlength{\itemsep}{0pt}\setlength{\parskip}{0pt}}
\setcounter{secnumdepth}{-\maxdimen} % remove section numbering
\ifLuaTeX
  \usepackage{selnolig}  % disable illegal ligatures
\fi

\title{Practical 1}
\author{Diego Aguilar-Ramirez}
\date{2022-03-14}

\begin{document}
\maketitle

\textbf{NOTE:}This script is largely based on material provided (see
``00\_Prelim\_material'' and 01) by course to set up R for MR analyses
used in course.

\hypertarget{updates-r}{%
\section{Updates R}\label{updates-r}}

\begin{Shaded}
\begin{Highlighting}[]
\ControlFlowTok{if}\NormalTok{ (}\SpecialCharTok{!}\FunctionTok{require}\NormalTok{(}\StringTok{"installr"}\NormalTok{)) }\FunctionTok{install.packages}\NormalTok{(}\StringTok{"installr"}\NormalTok{)}
\NormalTok{installr}\SpecialCharTok{::}\FunctionTok{updater}\NormalTok{()}
\end{Highlighting}
\end{Shaded}

\hypertarget{installs-packages-required-for-course-all-throughout}{%
\section{Installs packages required for course all
throughout}\label{installs-packages-required-for-course-all-throughout}}

\begin{Shaded}
\begin{Highlighting}[]
\ControlFlowTok{if}\NormalTok{ (}\SpecialCharTok{!}\FunctionTok{require}\NormalTok{(}\StringTok{"pacman"}\NormalTok{)) }\FunctionTok{install.packages}\NormalTok{(}\StringTok{"pacman"}\NormalTok{)}
\NormalTok{pacman}\SpecialCharTok{::}\FunctionTok{p\_load}\NormalTok{(ivpack, meta, MendelianRandomization)}
\end{Highlighting}
\end{Shaded}

\hypertarget{load-the-data}{%
\section{Load the data}\label{load-the-data}}

Data was downloaded from Moodle\\
\textbf{NOTE:} Although Rproj is at different wd (which is in version
control), I will set the wd to the directory where I downloaded the data
(which is not in version control) - might change later on.

\begin{Shaded}
\begin{Highlighting}[]
\NormalTok{coursedata }\OtherTok{=} \FunctionTok{read.csv}\NormalTok{(ROOTDIR }\SpecialCharTok{\%\&\%} \StringTok{"coursedata.csv"}\NormalTok{) }\CommentTok{\#Load data}
\FunctionTok{attach}\NormalTok{(coursedata) }\CommentTok{\#Attach coursedata to the R search path }
\end{Highlighting}
\end{Shaded}

\hypertarget{explore-the-data}{%
\section{Explore the data}\label{explore-the-data}}

\begin{Shaded}
\begin{Highlighting}[]
\FunctionTok{str}\NormalTok{(coursedata)}
\end{Highlighting}
\end{Shaded}

\begin{verbatim}
## 'data.frame':    1000 obs. of  8 variables:
##  $ ID   : int  1 2 3 4 5 6 7 8 9 10 ...
##  $ g1   : int  1 0 1 0 0 2 1 1 1 2 ...
##  $ g2   : int  0 0 1 0 0 0 0 0 0 0 ...
##  $ g3   : int  0 0 0 0 0 0 0 0 0 0 ...
##  $ g4   : int  1 0 1 0 1 0 0 0 0 0 ...
##  $ x    : num  0.5114 -1.0545 -0.563 0.8615 0.0156 ...
##  $ y    : num  -1.297 0.912 1.279 -1.438 2.719 ...
##  $ y.bin: int  1 0 0 1 0 0 1 1 0 0 ...
\end{verbatim}

\begin{Shaded}
\begin{Highlighting}[]
\FunctionTok{head}\NormalTok{(coursedata)}
\end{Highlighting}
\end{Shaded}

\begin{verbatim}
##   ID g1 g2 g3 g4           x          y y.bin
## 1  1  1  0  0  1  0.51141400 -1.2970720     1
## 2  2  0  0  0  0 -1.05450500  0.9117613     0
## 3  3  1  1  0  1 -0.56296820  1.2791060     0
## 4  4  0  0  0  0  0.86149410 -1.4378610     1
## 5  5  0  0  0  1  0.01561589  2.7192830     0
## 6  6  2  0  0  0  0.68330030  1.6415770     0
\end{verbatim}

\hypertarget{causal-estimate-using-the-ratio-method-for-a-continuous-outcome}{%
\section{1. Causal estimate using the ratio method for a continuous
outcome}\label{causal-estimate-using-the-ratio-method-for-a-continuous-outcome}}

The causal effect of the risk factor \texttt{x} on the continuous
outcome \texttt{y}can be estimated as the per allele genetic association
with the outcome \(\hat{\beta}_{Y_{j}}\) divided by the per allele
genetic association with the risk factor \(\hat{\beta}_{X_{j}}\) (the
ratio method):
\[ratio\ estimate\ for\ variant\ j = \frac{\hat{\beta}_{Y_{j}}}{\hat{\beta}_{X_{j}}}\]
\vspace{6pt}

\begin{Shaded}
\begin{Highlighting}[]
\NormalTok{by1 }\OtherTok{=} \FunctionTok{lm}\NormalTok{(y}\SpecialCharTok{\textasciitilde{}}\NormalTok{g1)}\SpecialCharTok{$}\NormalTok{coef[}\DecValTok{2}\NormalTok{] }\CommentTok{\#Genetic association with the outcome}
\NormalTok{bx1 }\OtherTok{=} \FunctionTok{lm}\NormalTok{(x}\SpecialCharTok{\textasciitilde{}}\NormalTok{g1)}\SpecialCharTok{$}\NormalTok{coef[}\DecValTok{2}\NormalTok{] }\CommentTok{\#Genetic association with the exposure}
\end{Highlighting}
\end{Shaded}

So, simply divide \texttt{by1} by \texttt{bx1} to calculate the ratio

\begin{Shaded}
\begin{Highlighting}[]
\NormalTok{beta.ratio }\OtherTok{=}\NormalTok{ by1}\SpecialCharTok{/}\NormalTok{bx1}
\NormalTok{beta.ratio }\CommentTok{\#Ratio estimate for g1}
\end{Highlighting}
\end{Shaded}

\begin{verbatim}
##          g1 
## -0.06302634
\end{verbatim}

The standard error of the causal estimate can be calculated simply as
the standard error of the genetic association with the outcome
\(se(\hat{\beta}_{Y_{j}})\) divided by the genetic association with the
risk factor \(\hat{\beta}_{X_{j}}\). This is the simplest form of the
standard error, and is the first-order term from a delta method
expansion for the standard error of a ratio (whatever this means, took
verbatim from practical material).
\[standard\ error\ of\ ratio\ estimate\ (first\ order) = \frac{se(\hat{\beta}_{Y_{j}})}{\hat{\beta}_{X_{j}}}\]
\vspace{6pt}

Practical asks to \emph{``calculate the first order standard error of
the ratio estimate for the first genetic variant \texttt{g1}.''} This
code should help:

\begin{Shaded}
\begin{Highlighting}[]
\NormalTok{byse1 }\OtherTok{=} \FunctionTok{summary}\NormalTok{(}\FunctionTok{lm}\NormalTok{(y}\SpecialCharTok{\textasciitilde{}}\NormalTok{g1))}\SpecialCharTok{$}\NormalTok{coef[}\DecValTok{2}\NormalTok{,}\DecValTok{2}\NormalTok{] }\CommentTok{\#Standard error of the G{-}Y association}
\NormalTok{se.ratio1first }\OtherTok{=}\NormalTok{ byse1}\SpecialCharTok{/}\FunctionTok{sqrt}\NormalTok{(bx1}\SpecialCharTok{\^{}}\DecValTok{2}\NormalTok{)  }
\NormalTok{se.ratio1first }\CommentTok{\#Standard error (first order) of the ratio estimate }
\end{Highlighting}
\end{Shaded}

\begin{verbatim}
##        g1 
## 0.6451987
\end{verbatim}

\emph{Since the estimate of \texttt{bx1} is not guaranteed to be
positive, we take the square, and then the square root of \texttt{bx1},
when calculating the first order standard error of the ratio estimate.
The standard error will not make sense if \texttt{bx1} is negative.}

The above approximation does not account for the uncertainty in the
denominator of the ratio estimate. This can be taken into account using
the second term of the delta method expansion:
\[standard\ error\ of\ ratio\ estimate\ (second\ order) =  \sqrt{\frac{se(\hat{\beta}_{Y_{j}})^2}{{\hat{\beta}_{X_{j}}}^2} + \frac{{\hat{\beta}_{Y_{j}}}^2{se(\hat{\beta}_{X_{j}})}^2}{{\hat{\beta}_{X_{j}}}^4}} \]
\vspace{6pt} Then calculate the second order standard error of the ratio
estimate for the first genetic variant \texttt{g1}. \vspace{48pt}

\begin{Shaded}
\begin{Highlighting}[]
\NormalTok{bxse1 }\OtherTok{=} \FunctionTok{summary}\NormalTok{(}\FunctionTok{lm}\NormalTok{(x}\SpecialCharTok{\textasciitilde{}}\NormalTok{g1))}\SpecialCharTok{$}\NormalTok{coef[}\DecValTok{2}\NormalTok{,}\DecValTok{2}\NormalTok{] }\CommentTok{\#Standard error of the G{-}X association}
\NormalTok{se.ratio1second }\OtherTok{=} \FunctionTok{sqrt}\NormalTok{(byse1}\SpecialCharTok{\^{}}\DecValTok{2}\SpecialCharTok{/}\NormalTok{bx1}\SpecialCharTok{\^{}}\DecValTok{2} \SpecialCharTok{+}\NormalTok{ by1}\SpecialCharTok{\^{}}\DecValTok{2}\SpecialCharTok{*}\NormalTok{bxse1}\SpecialCharTok{\^{}}\DecValTok{2}\SpecialCharTok{/}\NormalTok{bx1}\SpecialCharTok{\^{}}\DecValTok{4}\NormalTok{)}
\NormalTok{se.ratio1second }\CommentTok{\#Standard error (second order) of the ratio estimate}
\end{Highlighting}
\end{Shaded}

\begin{verbatim}
##        g1 
## 0.6459625
\end{verbatim}

The F-statistic from the regression of the risk factor on the genetic
variant(s) is used as a measure of `weak instrument bias', with smaller
values suggesting that the estimate may suffer from weak instrument
bias. Calculate the F-statistic for the first genetic variant
\texttt{g1}.

\begin{Shaded}
\begin{Highlighting}[]
\NormalTok{fstat1 }\OtherTok{=} \FunctionTok{summary}\NormalTok{(}\FunctionTok{lm}\NormalTok{(x}\SpecialCharTok{\textasciitilde{}}\NormalTok{g1))}\SpecialCharTok{$}\NormalTok{f[}\DecValTok{1}\NormalTok{]}
\NormalTok{fstat1}
\end{Highlighting}
\end{Shaded}

\begin{verbatim}
##    value 
## 4.027979
\end{verbatim}

\textbf{NOTE:}\emph{Care should be taken when interpreting the
F-statistic from observed data. Some studies recommend excluding genetic
variants if they have a F-statistic less than 10. Using such a stringent
cut-off may introduce more bias than it prevents, as the estimated
F-statistic can show considerable variation and may not provide a good
indication of the true stength of the instrument.}

The Minor Allele Frequency (MAF) is the frequency at which the second
most common allele occurs in a given population. Calculate the MAF for
\texttt{g1}, remembering that some people may have 2 copies of the
allele. \emph{``Total up the total number of alleles in the population
and divide by two times the size of the population''}

\begin{Shaded}
\begin{Highlighting}[]
\NormalTok{MAF }\OtherTok{=}\NormalTok{ (}\FunctionTok{sum}\NormalTok{(g1}\SpecialCharTok{==}\DecValTok{1}\NormalTok{) }\SpecialCharTok{+} \DecValTok{2}\SpecialCharTok{*}\FunctionTok{sum}\NormalTok{(g1}\SpecialCharTok{==}\DecValTok{2}\NormalTok{))}\SpecialCharTok{/}\NormalTok{(}\DecValTok{2}\SpecialCharTok{*}\FunctionTok{length}\NormalTok{(g1))}
\NormalTok{MAF}
\end{Highlighting}
\end{Shaded}

\begin{verbatim}
## [1] 0.3015
\end{verbatim}

Read in \texttt{summarized\_data.csv} to obtain the values that have
been calculated (for the course) for the other 3 genetic variants.

\begin{Shaded}
\begin{Highlighting}[]
\NormalTok{ratio.all}\OtherTok{\textless{}{-}}\FunctionTok{as.matrix}\NormalTok{(}\FunctionTok{read.csv}\NormalTok{(ROOTDIR }\SpecialCharTok{\%\&\%} \StringTok{"summarized\_data.csv"}\NormalTok{, }\AttributeTok{row=}\DecValTok{1}\NormalTok{)) }
\NormalTok{ratio.all}
\end{Highlighting}
\end{Shaded}

\begin{verbatim}
##                           g1         g2         g3         g4
## by              -0.008550605  0.2565551  0.2778433 0.17188563
## byse             0.087532274  0.1325076  0.1316129 0.12653807
## bx               0.135667161  0.4938326  0.3475786 0.06788043
## bxse             0.067597577  0.1015321  0.1014761 0.09798360
## beta.ratio      -0.063026340  0.5195184  0.7993682 2.53218242
## se.ratio.first   0.645198685  0.2683249  0.3786566 1.86413185
## se.ratio.second  0.645962479  0.2888032  0.4447983 4.10305056
## fstat            4.027979155 23.6566409 11.7321775 0.47993491
## MAF              0.301500000  0.1010000  0.1030000 0.11150000
\end{verbatim}

\end{document}
